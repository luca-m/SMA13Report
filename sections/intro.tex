\section{Introduction}
\label{intro}

In the last ten years we have assisted to the explosion of new pervasive technologies which have changed the way we intend computing, modified our personal and social habits, and consequentially opened new opportunities and research branch.\\
Pervasive computing has been matter of researches by both academia and industry, so a good amount of literature has been published during this period; in a synthetic fashion we can extrapolate that we are facing some kind of open, distributed, and possibly heterogeneous complex-system which have to achieve a goal without a centralized coordination. For this reason developing algorithms and solutions that run on top of this kind of systems are considered at least challenging by software engineers, in fact the difficulty to apply formal methods for validating this kind of systems have enhanced the simulation's role inside development processes. \\
Having said that, in this class project we aim to exploit this important tool in the software engineer arsenal (simulation) for setting up an experiment related to pervasive computing.\\
Before describe the experiment we have to provide the reader a bit more information which helps to contextualize it. \\
Background theme of this project is the social-based forwarding in opportunistic network, so, broadly speaking, we are facing the problem of delivery or acquire a content in an infrastructure-free scenario composed by a multitude of mobile-device (typically carried by humans) by exploiting what we know about social relationship between devices/humans with respect to their movements and their consequential physical contacts. \\
This class of problems were one of the topics in \emph{SOCIALNETS: Social networking for pervasive adaptation}\cite{socialnetseu}: an European FP7-ICT research project which investigated this kind of problems and produced several prototypes spendable in infrastructure-free scenario. \\
In particular, we are focusing in a subtopic treated during this project: algorithms for social-based forwarding in opportunistic, delay tolerant network. \\
Several algorithms have been evaluated during the course of SOCIALNETS and an interesting fact we have noted is that performed experiments were mainly based on emulation approaches\cite{bubble} based on contact network data collected during \emph{Haggle Project}\footnote{\url{http://www.haggleproject.org}}.\\
Due to the notorious difficulty of collecting this kind of data, in fact collecting process involved lot of people and required from days to years, we consider interesting to experiment this kind of algorithm in a simulated environment, trying to adopt proper mobility models in order to obtain realistic performance measurement.\\
In the following sections we'll first review mayor social network topologies\ref{social_networks}, which are exploited by forwarding algorithms, on mobility models\ref{mobility_models} that enable to reproduce realistic contact network dynamics, even using social network informations, and also a review on social-based forwarding algorithm\ref{forwarding} with a focus on \emph{Bubble} algorithms\cite{bubble}, because they shown better performance in emulations performed during SOCIALNETS project, terminating with the description and the results of the class project experiment\ref{experiment}. \\ 

