\section{The Experiment}
\label{experiment}

%Occorre utilizzare un simulatore che supporti sia i concetti di mobilità
%che 
%ne di topologia sociale in maniera da modellare correttamente il dominio
%di riferimento (nodi mobili, rete opportunistica e rete sociale).

\subsection{Simulation platform}
\label{exp_platform}

%Il punto di partenza che è stato al momento individuato prevede l'utilizzo
%di /Alchemist/ come base per le simulazioni. Il simulatore supporta già 
%mobilità dei nodi e generazione di rete opportunistica. Inoltre può 
%risultare interessante la metafora chimica per la definizione di 
%regole di comportamento locali.

\subsection{Platform incarnation and algorithms}
\label{exp_incarnation}

% 2. Supporto a generazione di reti random e/o scale-free (quindi
%    supporto a simulazioni su tali topologie) 
% 3. Supporto di ambienti ibridi: nodi mobili con relazioni di vicinato
%    opportunistiche insieme a relazioni sociali

\subsection{Experiment setup}
\label{exp_setup}

% 4. realizzazione di mobility model basato su community
% 5. Simulazione di un algoritmo "bubble rap -like", misura di delivery
%    ratio e delivery cost


\subsection{Results}
\label{exp_results}


