\section{Routing in opportunistic networks}
\label{routing}

%TODO little introduction

\subsection{Non social-based algorithms}
\label{r_non_social}

% WAIT: Hold onto a message until the sender encounters the recipient directly, which represents the lower bound
% FLOOD: Messages are flooded throughout the entire system, which represents the upper bound for delivery
% Multiple-Copy-multiple-hoP (MCP): 

\subsection{social-based algorithms}
\label{r_social}
% LABEL: Explicit labels are used to identify forwarding nodes that belong to the same organisation.

% RANK: The forwarding metric used in this algorithm is the node centrality.

% DEGREE: The forwarding metric used in this algorithm is the node degree

% BUBBLE: The BUBBLE family of protocols combines the observed hierarchy of centrality of nodes and observed

\subsubsection{Bubble algorithm}
\label{r_bubble}

%I parametri degli esperimenti usati sono stati: 
%
%  * Numero di copie: numero di duplicati per ogni messaggio creati ad
%    ogni nodo. 
%  * Numero di Hops 
%  * TTL 
%
%E per ogni emulazione condotta sono state definite metriche
%per misurarne efficacia ed efficienza:
%
%  * Delivery Ratio: messaggi arrivati a destinazione (msg unici) 
%  * Delivery Cost: numero di messaggi (duplicati inclusi) trasmessi 
%    (normalizzato con numero di msg unici) 